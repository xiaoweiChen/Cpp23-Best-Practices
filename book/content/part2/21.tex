平台上支持至少两种编译器。每种编译器执行不同的分析,并以略微不同的方式实现标准。

如果使用 Visual Studio,应该能够相对轻松地在 clang 和 cl.exe 之间切换,也可以使用 WSL 并启用远程 Linux 构建。

如果使用 Linux,应该能够轻松地在 GCC 和 Clang 之间切换。

\begin{myNotic}{}
在 MacOS 上,请确保使用的编译器确实是您认为的那个。gcc 命令很可能是 Apple 安装的 clang 的符号链接。
\end{myNotic}

\begin{myNotic}{}
apple-clang 并不等同于主线版本的 clang,其版本号也不一致。通常很难知道 apple-clang 支持哪些特性。cppreference 的编译器支持参考可能会有所帮助\footnote{\url{https://en.cppreference.com/w/cpp/compiler_support}}。
\end{myNotic}

要在平台上安装更新或不同的编译器,可参考以下资源:

Ubuntu / Debian

\begin{itemize}
\item
GCC - Toolchain PPA\footnote{\url{https://launchpad.net/~ubuntu-toolchain-r/+archive/ubuntu/ppa}}

\item
Clang - apt 包\footnote{\url{https://apt.llvm.org/}}

\item
GCC MinGW\footnote{\url{http://mingw.org/}}

\item
Clang 官方下载\footnote{\url{https://releases.llvm.org/download.html}}
\end{itemize}

MacOS

\begin{itemize}
\item
Homebrew / MacPorts
\end{itemize}

\begin{myTip}{练习:添加另一个编译器}
既然已经启用了系统的持续构建,现在是时候添加另一个编译器了。

升级到当前所用编译器的新版本总是一个好主意。但如果只支持 GCC,请考虑添加 Clang;或者如果只支持 Clang,则添加 GCC。如果使用 Windows,请在 MSVC 之外添加 MinGW 或 Clang。
\end{myTip}

\begin{myTip}{练习:添加另一个操作系统}
当希望项目至少有一部分可以移植到另一个操作系统上。尝试让项目的一部分在另一个操作系统和工具链上成功编译,将有助于了解当前的代码。
\end{myTip}

\mySubsectionNoFile{21.1}{资源}

\begin{itemize}
\item
C++Now 2015: Jason Turner “编写可移植代码:如何/为何让 C++ 项目跨平台”\footnote{\url{https://youtu.be/cb3WIL96N-o}}
\end{itemize}