如果没有自动化测试,就不可能维持项目的质量。

在我整个职业生涯所参与的 C++ 项目中,我不得不支持以下多种组合:

\begin{itemize}
\item
x86

\item
x64

\item
SPARC

\item
ARM

\item
MIPSEL
\end{itemize}

运行于以下操作系统:

\begin{itemize}
\item
Windows

\item
Solaris

\item
MacOS

\item
Linux
\end{itemize}

当你在多个平台和架构上使用多种编译器组合时,某一平台上的重大改动极有可能会破坏一个或多个其他平台的构建或功能。

为了解决这个问题,请为你的项目启用持续构建和持续测试。

\begin{itemize}
\item
测试你所支持的所有可能的平台组合

\item
分别测试 Debug 和 Release 构建

\item
测试所有配置选项

\item
使用比你当前支持或要求的更新版本的编译器进行测试
\end{itemize}

\begin{myNotic}{}
Sanitizer(检测工具)只能对未被优化器移除的代码进行插桩,而优化器有时会利用某些未定义行为(UB)进行优化。因此,对于至少一个你支持的平台,你必须同时具备以下构建配置:Debug、Debug + Sanitizers、Release、Release + Sanitizers!
\end{myNotic}

\begin{myWarning}{警告}
如果你不要求 100\% 的测试通过率,你将永远无法真正了解代码的实际状态。
\end{myWarning}

\begin{myTip}{练习:启用持续构建}
了解你所在组织当前的持续构建环境。如果尚未建立,阻碍其搭建的障碍是什么?为你的项目搭建类似 GitLab CI、GitHub Actions、Appveyor 或 Travis 这样的系统有多困难?
\end{myTip}