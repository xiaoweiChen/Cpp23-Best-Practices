编译警告中大多数警告实际上非常有益。在 GCC 和 Clang 中,-Wall 并不会开启所有警告,-Wextra 也仍然只是触及了表面而已!

\begin{myNotic}{}
MSVC 上的 /Wall 才是真正开启所有警告。我们的编译器开发人员不建议在 MSVC 上使用 /Wall 或在 Clang 上使用 -Weverything,因为其中许多是诊断性警告,并不需要采取操作。GCC 并未提供等效选项。
\end{myNotic}

强烈建议使用 -Wpedantic(GCC/Clang)。该命令行选项会禁用语言扩展,使代码更贴近 C++ 标准。你现在启用的警告越多,将来移植到其他平台时就会越轻松。

\begin{myNotic}{}
从 MSVC 的 C++20 模式开始,/permissive- 已不再需要,它现在是 cl.exe 的默认设置。
\end{myNotic}

\begin{myTip}{练习:启用更多警告}
\begin{itemize}
\item
探索编译器支持的所有警告选项,尽可能多地启用。

\item
修复由此产生的新警告。

\item
回到第 1 步,重复此过程。
\end{itemize}

\begin{myNotic}{}
MSVC 提供了一组优秀的警告,可通过警告级别启用。你可以从 /W1 开始,随着逐步修复警告,逐步提升到 /W4。
\end{myNotic}

这个过程可能繁琐且无意义,但这些警告确实能捕捉到真实的 bug。

\end{myTip}

\begin{myTip}{练习:讨论在持续集成(CI)中启用 -Werror 或 -WX,以确保警告不会累积。}
我观察到,必须在开发人员本地机器和 CI 系统上都启用“警告视为错误”,否则开发人员会把 CI 当成敌人,并不断禁用越来越多的警告!
\end{myTip}

\mySubsectionNoFile{16.1}{资源}

\begin{itemize}
\item
C++ Best Practices 网站整理的推荐警告列表\footnote{\url{https://github.com/lefticus/cppbestpractices/blob/master/02-Use_the_Tools_Available.md\#compilers}}

\item
GCC 完整警告列表\footnote{\url{https://gcc.gnu.org/onlinedocs/gcc/Warning-Options.html}}

\item
Clang 完整警告列表\footnote{\url{https://clang.llvm.org/docs/DiagnosticsReference.html}}

\item
MSVC 默认关闭的编译器警告\footnote{\url{https://docs.microsoft.com/en-us/cpp/preprocessor/compiler-warnings-that-are-off-by-default?view=vs-2019}}

\item
C++ Weekly 第 168 集 - 发现该使用的警告\footnote{\url{https://youtu.be/IOo8gTDMFkM}}
\end{itemize}