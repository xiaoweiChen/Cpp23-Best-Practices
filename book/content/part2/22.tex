您的想象力决定了您能创建的测试的边界。您在调用自己的 API 时,是否尝试过带有恶意的目的?是否会故意向输入传递格式错误的数据?是否会处理来自未知或不可信来源的输入?

要生成所有可能的函数调用、所有可能的输入以及所有可能的组合,这是不可能完成的任务。幸运的是,已有工具可以解决这一问题。

\mySubsectionNoFile{22.1}{模糊测试(Fuzzing)}

模糊测试器会生成各种长度的随机数据流。您编写的测试框架会接收这些数据流,并以适合您应用程序的方式对其进行处理。模糊测试器会分析测试执行过程中生成的代码覆盖率数据,并利用这些信息剔除冗余的测试,同时生成新的、具有新颖性和独特性的测试用例。

理论上,如果运行时间足够长,模糊测试最终能够达到被测代码 100\% 的代码覆盖率。结合 AddressSanitizer 使用时,它将成为发现代码缺陷的强大工具。一篇发表于 2015 年的有趣文章\footnote{\url{https://blog.hboeck.de/archives/868-How-Heartbleed-couldve-been-found.html}}指出,若将模糊测试工具与 AddressSanitizer 结合使用,原本可以在不到 6 小时内发现 OpenSSL 中“心脏出血”(heartbleed)的安全漏洞。

\begin{myNotic}{}
这个 6 小时的说法如今已严重过时。借助现代模糊测试工具和更先进的计算机,像 heartbleed 这样的漏洞只需几分钟即可被发现。
\end{myNotic}

\begin{myTip}{}
模糊测试主要用于发现内存安全和安全性方面的缺陷。
\end{myTip}

目前存在多种不同的模糊测试工具。为便于本节说明,我将重点介绍 LLVM 的 libFuzzer\footnote{\url{https://www.llvm.org/docs/LibFuzzer.html}}。大多数模糊测试工具的基本原理是相同的。

您必须提供某种形式的入口点。该入口点通常是一个如下所示的函数:

\filename{图 7. libFuzzer 入口点}

\begin{cpp}
extern "C" int LLVMFuzzerTestOneInput(const uint8_t *Data, 
                                      size_t Size);
\end{cpp}

其中,Data 指针始终有效,Size 参数大于或等于 0。

如果您的库主要负责解析输入文件(例如 libpng),那么您的工作就非常简单:

\filename{图 8. 使用 libFuzzer 数据}

\begin{cpp}
extern "C" int LLVMFuzzerTestOneInput(const uint8_t *Data,
                                      size_t Size)
{
  parseInput(Data, Size);
}
\end{cpp}

如果您的函数接收的是数据结构而非输入字符串,那么您的工作会稍微复杂一些,但仍然可行。

\filename{图 9. 高级 libFuzzer 数据用法}

\begin{cpp}
template<typename Type>
std::tuple<const uint8_t *, size_t, Type>
    createStruct(const uint8_t *Data, size_t Size)
{
  // we're only allowed to do this with trivial types
  static_assert(std::is_trivial_v<Type>);
  Type result{}; // default initialize
  const auto bytesToRead = std::min(sizeof(Type), Size);
  std::memcpy(&result, Data, bytesToRead);
  return {std::next(Data, bytesToRead), Size - bytesToRead, result};
}

extern "C" int LLVMFuzzerTestOneInput(const uint8_t *Data,
                                      size_t Size)
{
  // This example is meant as inspiration, it has not been
  // tested in a real test
  auto [newDataPtr, remainingSize, Obj1]
    = createStruct<Type1>(Data,Size);
  auto [lastDataPtr, lastSize, Obj2]
    = createStruct<Type2>(newDataPtr, remainingSize);

  functionToTest(Obj1, Obj2);
}
\end{cpp}

模糊测试器很快就会发现,任何输入数据长度 Size > sizeof(Type1) + sizeof(Type2) 的情况都不会产生新的代码路径,因此它会集中精力生成合适长度的数据。

\begin{myTip}{}
可以了解一下 Google 在 2022 年推出的新型模糊测试库 FuzzTest\footnote{\url{https://github.com/google/fuzztest}},该库旨在简化将模糊测试集成到项目中的流程。
\end{myTip}

\mySubsectionNoFile{22.2}{变异测试}

变异测试通过修改被测代码中的条件判断和常量来工作。

\filename{图 10. 伪代码示例}

\begin{cpp}
bool greaterThanFive(const int value) {
  return value > 5; // comparison
}

void tests() {
  assert(greaterThanFive(6));
  assert(!greaterThanFive(4));
}
\end{cpp}

一个变异测试工具可能会修改常量 5 或者操作符 >,使得代码变为:

\filename{图 11. 变异后的代码}

\begin{cpp}
bool greaterThanFive(const int value) {
  return value < 5; // mutated
}
\end{cpp}

如果某个变异后的代码仍然能通过所有测试,那么这个未被“杀死”的变异体(survived mutant)可能表明测试本身存在缺陷,或者原代码中存在潜在的 bug。

\begin{myTip}{练习:创建一个模糊测试框架}
应用本节中的示例,为您的代码创建模糊测试器。您遇到了哪些挑战?
\end{myTip}

\begin{myTip}{练习:研究变异测试}
本书作者没有直接使用过变异测试的经验。它是否适用于您的项目?您能找到哪些有趣的资源?
\end{myTip}

\begin{myTip}{练习:如果可以,请尝试 FuzzTest}
FuzzTest 目前仅支持非常新版本的 clang,请查阅当前支持的编译器列表,看看是否可以在您的代码中尝试使用它。
\end{myTip}

\mySubsectionNoFile{22.3}{资源}

\begin{itemize}
\item
C++Now 2018: Marshall Clow “使用模糊测试让您的库更可靠”\footnote{\url{https://youtu.be/LlLJRHToyUk}}

\item
C++ Weekly Ep 85: 模糊测试\footnote{\url{https://youtu.be/gO0KBoqkOoU}}

\item
CppCast: Alex Denisov “使用 Mull 进行变异测试”\footnote{\url{https://cppcast.com/alex-denisov/}}

\item
NDC TechTown 2019: Seph De Busser “测试测试:C++ 中的变异测试”\footnote{\url{https://youtu.be/M-5_M8qZXaE}}

\item
CppCon 2017: Kostya Serebryany “不模糊测试,就失败……”\footnote{\url{https://youtu.be/k-Cv8Q3zWNQ}}

\item
CppCon 2020: Barnabás Bágyi “对类接口进行模糊测试以生成和运行 libFuzzer 测试”\footnote{\url{https://cppcon2020.sched.com/event/e7An/fuzzing-class-interfaces-for-generating-and-running-testswith-libfuzzer?iframe=no}} —— 关于以新颖方式使用模糊测试的启发性演讲。该视频目前尚未上传至 YouTube,但本书出版后请留意查找。

\item
Autotest\footnote{\url{https://gitlab.com/wilzegers/autotest/}} —— 与“对类接口进行模糊测试以生成和运行 libFuzzer 测试”演讲相关的库。

\item
FuzzTest\footnote{\url{https://github.com/google/fuzztest}} —— Google 提供的模糊测试库。

\item
DeepState\footnote{\url{https://github.com/trailofbits/deepstate}} —— TrailOfBits 提供的模糊测试工具。

\item
oss-fuzz\footnote{\url{https://github.com/google/oss-fuzz}} —— 针对开源项目的持续模糊测试服务。

\item
Mutate++\footnote{\url{https://github.com/nlohmann/mutate_cpp}} —— 一个 C++ 变异测试工具。
\end{itemize}











