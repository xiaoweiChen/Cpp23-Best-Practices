最佳实践,简单来说,就是关于:

\begin{enumerate}
\item
减少常见错误

\item
快速发现错误

\item
在不牺牲(并且通常还能提升)性能的前提下实现以上两点
\end{enumerate}

\mySubsectionNoFile{3.1}{为何需要最佳实践?}

首先,让我们明确一点:

\mySubsubsection{3.1.1}{你的项目并不特殊}

如果你在使用 C++ 编程,那么你或你公司的某个人一定在乎性能。否则,他们可能早就选择使用其他编程语言了。我接触过许多公司,他们都告诉我自己很特殊,因为他们需要快速完成任务!

剧透一下:他们做出的决定其实都大同小异,原因也基本相同。真正例外的很少。那些做出不同决策的特例公司,恰恰是已经遵循了本书建议的组织。

\mySubsectionNoFile{3.2}{最坏的情况是什么?}

我不想制造悲观情绪,但让我们花点时间思考一下:如果你的项目存在严重缺陷,最坏的情况会怎样?

\noindent
\textbf{游戏}

严重缺陷会导致远程漏洞或攻击入口。

\noindent
\textbf{金融}

严重缺陷会导致巨额资金损失、交易加速异常、甚至市场崩盘\footnote{\url{https://en.wikipedia.org/wiki/2010_flash_crash}}。

\noindent
\textbf{航空航天}

严重缺陷会导致航天器损毁或人员伤亡\footnote{\url{https://spectrum.ieee.org/aerospace/aviation/how-the-boeing-737-max-disaster-looks-to-a-softwaredeveloper}}。

\noindent
\textbf{你的行业}

严重缺陷会导致……资金损失?人员失业?远程被黑?还是更糟的情况?

\mySubsectionNoFile{3.3}{示例}

本书中的所有示例都使用 struct 而非 class。struct 和 class 唯一的区别在于,struct 的成员和基类默认为 public。使用 struct 可以让示例更简短、更易于阅读。

\mySubsectionNoFile{3.4}{练习}

每个章节都包含一个或多个练习。大多数练习没有绝对正确或错误的答案。

\begin{myTip}{练习:留意练习题}
在接下来的章节中,你会看到像这样的练习题。请留意它们!

这些练习具有以下特点:

\begin{itemize}
\item
注重实践性,可直接应用到你当前的代码库中,立即体现其价值。

\item
鼓励你进行独立思考和研究,从而更深入地理解这门语言。
\end{itemize}
\end{myTip}

\mySubsectionNoFile{3.5}{链接与参考}

我已尽力引用那些我曾向其学习的人,并尽可能附上他们演讲的链接。如果我遗漏了什么,请告诉我。





