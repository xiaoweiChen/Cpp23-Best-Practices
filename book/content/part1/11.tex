好吧,会导致未定义行为(Undefined Behavior, UB)的情况有很多,很难一一记住,因此我们将在接下来的章节中给出一些示例。你最需要理解的关键一点是:一旦存在未定义行为,你的整个程序就可能崩溃或产生不可预测的结果。

[intro.abstract\footnote{\url{http://eel.is/c++draft/intro.compliance\#intro.abstract-5}}]

\begin{verbatim}
    一个符合标准的实现,在执行一个格式良好(well-formed)的程序时,其
    产生的可观察行为,应当与该程序在对应的抽象机器实例上
    (使用相同程序和相同输入)的某次可能执行结果一致。

    然而,如果任何此类执行中包含未定义操作,
    那么本标准对实现执行该程序和输入的行为不作任何要求
    (甚至对第一个未定义操作之前的操作也不作要求)。
\end{verbatim}

请注意这句话:“本标准对实现执行该程序和输入的行为不作任何要求(甚至对第一个未定义操作之前的操作也不作要求)”。  

这意味着,一旦你的程序中存在未定义行为,整个程序都变得不可信。

\begin{myNotic}{}
接下来的几项内容将讨论如何降低项目中出现未定义行为的风险。
\end{myNotic}

\begin{myTip}{练习:使用 UBSan、ASan 和编译警告}
要完全理解所有未定义行为几乎是不可能的。幸运的是,我们有一些工具可以帮助我们。希望你已经在代码中启用了 UBSan(未定义行为 sanitizer)、ASan(地址 sanitizer)以及各类编译警告。现在正是回顾并评估你已启用的选项的好时机,看看是否能发现任何新的可用工具或选项。
\end{myTip}

\mySubsectionNoFile{11.1}{资源}

\begin{itemize}
\item
C++Now 2018: John Regehr “闭幕主题演讲:未定义行为与编译器优化”\footnote{\url{https://youtu.be/AeEwxtEOgH0}}

\item
CppCon 2018: Barbara Geller \& Ansel Sermersheim “未定义行为不是错误”\footnote{\url{https://youtu.be/XEXpwis_deQ}}
\end{itemize}