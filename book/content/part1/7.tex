我不是第一个提出这一点的人,也不会是最后一个。我认为这个概念现在已经广为接受,但我仍然看到一些 C++ 初学者过于专注于“面向对象编程(OOP)”。

Bjarne Stroustrup 在《C++ 程序设计语言》第三版中指出:

\begin{verbatim}
    C++ 是一种通用编程语言,偏向于系统编程,它:

    • 是更好的 C,
    • 支持数据抽象,
    • 支持面向对象编程,
    • 支持泛型编程。
\end{verbatim}

你必须理解,C++ 是一门多范式的编程语言,只有这样,才能充分发挥这门语言的优势。C++ 有效地支持当今存在的几乎所有编程范式。

\begin{itemize}
\item
过程式

\item
函数式

\item
面向对象

\item
泛型

\item
编译时编程(constexpr 和模板元编程)
\end{itemize}

知道何时恰当地使用这些工具中的每一种,才是编写优秀 C++ 代码的关键。那些僵化地固守单一范式的项目,往往会错失这门语言最优秀的特性。

\begin{myWarning}{警告}
不要总是试图同时使用所有可能的技术。那样最终只会得到一团难以维护和阅读的混乱代码。在合适的时机恰当地使用合适的技术,需要自律和实践。
\end{myWarning}

\begin{myTip}{练习:质疑你当前的设计}
如果你能跳出项目当前的设计模式,你会做出哪些不同的选择?
\end{myTip}

\mySubsectionNoFile{7.1}{资源}

\begin{itemize}
\item
《C++ 中的函数式编程》\footnote{\url{https://www.manning.com/books/functional-programming-in-c-plus-plus?a_aid=FPinCXX&a_bid=441f12cc}}

\item
C++ Weekly 第 137 集:C++ 不是一门面向对象语言\footnote{\url{https://youtu.be/AUT201AXeJg}}
\end{itemize}