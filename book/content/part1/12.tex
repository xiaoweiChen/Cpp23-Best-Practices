\filename{图 2. 对 'this' 是否为 'nullptr' 的无效检查}

\begin{cpp}
int Class::member() {
  if (this == nullptr) {
    // removed by the compiler, it would be UB
    // if this were ever null
    return 42;
  } else {
    return 0;
  }
}
\end{cpp}

从技术上讲,这个检查本身并不构成未定义行为(UB)。但这个检查永远不可能失败。如果 this 真的等于 nullptr,那么程序就已经处于未定义行为状态了。过去人们经常这样做,但这种行为一直都是未定义的。你不能在对象的生命周期之外访问它。如今的编译器总是会移除这种检查。

理论上唯一能让 this 为 nullptr 的情况,是你直接在一个空对象上调用成员函数。

\begin{myWarning}{警告}
接下来是反面教材,请勿模仿。
\end{myWarning}

\filename{图 3. 在 'nullptr' 上调用成员函数的错误示例。}

\begin{cpp}
Type *obj = nullptr;
obj->do_thing(); // never do this
\end{cpp}

即使是在 delete this 这种(技术上可行但绝不应做)的场景下也是如此。

\filename{图 4. 'delete this' 的错误示例}

\begin{cpp}
struct S {
  std::string data;

  void delete_yourself() {
    // do things
    delete this; // technically OK

    if (this) {
      // this block will always be executed, nothing changed
      // our view of `this`
    }

    // never do this
    data.size(); // UB, data's lifetime has ended
  }
};
\end{cpp}

在现代编译器上,if (this) 这样的检查永远不可能返回 false。

\begin{myTip}{练习:你会检查 this 是否为 nullptr 吗?}
对 nullptr 的检查可能以检查 NULL 或与 0 比较的形式隐藏。对 this 是否为 NULL 的检查很可能只存在于非常古老的代码库中。请确保你已启用编译警告,然后查找这些情况。

在你的代码库中搜索 this == 也是一个有趣的尝试,看看会发现哪些奇怪的代码。
\end{myTip}

\mySubsectionNoFile{12.1}{资源}

\begin{itemize}
\item
移植到 GCC-6:优化会移除对 this 的空指针检查\footnote{\url{https://www.gnu.org/software/gcc/gcc-6/porting_to.html\#this-cannot-be-null}}
\end{itemize}













