
栈对象(局部作用域内且未动态分配的对象)对优化器更友好、对缓存更友好,并且可能被优化器完全消除。正如 Björn Fahller 所说:“假设任何指针间接寻址都会导致缓存未命中。”

用最简单的术语来说:

\filename{图36. 好的想法,使用栈且可优化}

\begin{cpp}
std::string make_string() { return "Hello World"; }
\end{cpp}

\filename{图37. 糟糕的想法,使用了堆}

\begin{cpp}
std::unique_ptr<std::string> make_string() {
  return std::make_unique<std::string>("Hello World");
}
\end{cpp}

\filename{图38. 好的想法}

\begin{cpp}
void use_string() {
  // 此字符串位于栈上
  std::string value("Hello World");
}
\end{cpp}

\filename{图39. 非常糟糕的想法,使用了堆并且会泄漏内存}

\begin{cpp}
void use_string() {
  // 字符串位于堆上
  std::string *value = new std::string("Hello World");
}
\end{cpp}

\begin{myNotic}{}
注意,std::string 本身可能会在内部进行分配,从而使用堆。如果目标是完全不使用堆,那么需要采取其他措施。我们的目标是避免不必要的堆分配。
\end{myNotic}

一般来说,使用 new 表达式(或通过 make\_unique 或 make\_shared)创建的对象是堆对象,具有动态存储期。在局部作用域内创建的对象是栈对象,具有自动存储期。编译器和工具通常更容易发现对未初始化的栈值的读取,而非堆值。

\begin{myTip}{练习:查找堆的使用}
一些具有 C 和 Java 背景的开发者可能很难理解这一点。对于 Java 来说,因为创建对象需要使用 new。对于 C 语言来说,因为 C 编译器无法像 C++ 编译器那样执行相同类型的优化。因此,当前的代码中可能就存在一些不必要的堆使用。
\end{myTip}

\begin{myTip}{练习:运行堆分析器}
有多种堆分析工具,甚至 IDE 中可能就内置了一个。检查堆使用情况,寻找项目中对堆的潜在滥用。很可能大部分的堆分配都来自于对 std::string 或 std::vector 等容器的复制。
\end{myTip}

\mySubsectionNoFile{32.1}{资源}

\begin{itemize}
\item
Code::Dive 2018: Björn Fahller “缓存友好是什么意思?”\footnote{\url{https://youtu.be/Fzbotzi1gYs}}

\item
heaptrack - 一个用于 Linux 的堆内存分析器\footnote{\url{https://github.com/KDAB/heaptrack}}

\item
Massif: 一个堆分析器\footnote{\url{https://valgrind.org/info/tools.html\#massif}}
\end{itemize}



















