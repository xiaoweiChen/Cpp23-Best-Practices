没有析构函数在正确的情况下总是更好的。空的析构函数会破坏性能:

\begin{itemize}
\item
它们使类型不再平凡(trivial)

\item
没有任何功能用途

\item
可能影响析构的内联

\item
隐式地禁用了移动操作
\end{itemize}

\begin{myWarning}
如果你需要析构函数,是因为你在进行资源管理,或者在定义一个带有虚函数的基类,那么你需要遵循“五规则”。
\end{myWarning}

如果你提供了一个自定义删除器,std::unique\_ptr 可以帮助你应用“零规则”。

\begin{myTip}{练习:在你的代码中查找违反“零规则”的情况}

查找像这样的代码(我保证你一定能找到)。

\filename{图51. 空的、无意义的析构函数}

\begin{cpp}
struct S {
	// a bunch of other things
	~S() {}
};
\end{cpp}

或者更糟的情况:

\filename{图52. 前向声明的、空的、无意义的析构函数。}

\begin{cpp}
// file.hpp
struct S {
	~S();
}

// file.cpp
S::~S() {}
\end{cpp}
\end{myTip}

\begin{myNotic}
对特殊成员函数的任何提及都会隐式地禁用编译器生成的移动操作。这包括 ~S() = default;。
\end{myNotic}

这些析构函数是必要的吗?如果不是,请将它们移除。
如果这些析构函数存在于被多处使用的类型中,通过执行这个简单的操作,你很可能会测量到更小的二进制文件大小和更好的性能。

\begin{verbatim}
pImpl 惯用法的某些用法要求你定义一个析构函数。在这种情况下,请务必遵循“五规则”。
\end{verbatim}

\begin{myTip}{练习:使用compiler-explorer来查看违反“零规则”的代价。}
\filename{图53. “零规则”意外影响}

\begin{cpp}
// experiment with this C++20 example in
// compiler-explorer.com
#include <vector>
#include <string>

struct S
{
	std::string data;

	// uncomment this line and observe the size of the compiled
	// binary in both -O3 and -O0 builds.
	//~S() = default;
};

void some_func(std::vector<S> &data){
	data.emplace_back();
}
\end{cpp}
\end{myTip}

\mySubsectionNoFile{39.1}{资源}

\begin{itemize}
\item
C++ Reference: 三/五/零规则 \footnote{\url{https://en.cppreference.com/w/cpp/language/rule_of_three}}

\item
C++ Weekly 第154集:“减少代码膨胀的一个简单技巧”\footnote{\url{https://youtu.be/D8eCPl2zit4}}

\item
CppCon 2019: Jason Turner “伟大的 C++ is\_trivial”\footnote{\url{https://youtu.be/ZxWjii99yao}}
\end{itemize}








