考虑一下 POSIX 套接字的 API:

\filename{图 12. POSIX 套接字 API}

\begin{cpp}
socket(int, int, int);
\end{cpp}

这些参数(顺序不定)代表:

\begin{itemize}
\item
类型(type)

\item
协议(protocol)

\item
域(domain)
\end{itemize}

这种设计有问题,但类似的问题也可能潜藏在我们的代码中。

\filename{图 13. 定义不佳的构造函数}

\begin{cpp}
Rectangle(int, int, int, int);
\end{cpp}

这个函数的参数可能是 (x, y, width, height),也可能是 (x1, y1, x2, y2)。虽然可能性较小,但也可能是 (width, height, x, y)。

如何看待如下这样的 API 呢?

\filename{图 14. 强类型化的构造函数}

\begin{cpp}
Rectangle(Position, Size);
\end{cpp}

许多情况下,只需稍加努力,就可以创建出类型更强的 API。

\filename{图 15. 更强类型的定义}

\begin{cpp}
struct Position {
  int x;
  int y;
};

struct Size {
  int width;
  int height;
};

struct Rectangle {
  Position position;
  Size size;
};
\end{cpp}

这还能进一步支持通过运算符重载实现逻辑上可组合的表达式,例如:

\filename{图 16. 关联类型的运算符重载}

\begin{cpp}
// Return a new rectangle that has been
// moved by the offset amount passed in
Rectangle operator+(Rectangle, Position);
\end{cpp}

\begin{myTip}{}
某些情况下,创建结构体甚至可能提升性能,详见 C++ Weekly 第119集《负成本结构体》\footnote{\url{https://youtu.be/FwsO12x8nyM}}。
\end{myTip}

\mySubsectionNoFile{28.1}{避免使用布尔参数}

本书的一位预发布读者指出,Steve Maguire 在其著作《Writing Solid Code》(编写坚固的代码)第 5 章中提到:“让代码在调用点清晰可读。避免使用布尔参数。”

C++11 中的 enum class 提供了一种简便的方式,可以实现更强的类型检查,避免使用布尔参数,并使 API 更难误用。

来看以下代码:

\filename{图 17. 参数顺序不明确的构造函数}

\begin{cpp}
struct Widget {
  // this constructor is easy to use wrong, we
  // can easily transpose the parameters
  Widget(bool visible, bool resizable);
}
\end{cpp}

对比以下改进版本:

\filename{图 18. 使用作用域枚举实现更强的类型检查}

\begin{cpp}
struct Widget {
  enum struct Visible { True, False };
  enum struct Resizable { True, False };

  // still possible to use this wrong, but MUCH harder
  Widget(Visible visible, Resizable resizable);
}
\end{cpp}

\begin{myTip}{练习:识别现有代码中的问题 API}
您是否经常把某些函数调用的参数顺序搞错?该如何修复?
\end{myTip}

\begin{myTip}{练习:研究 C++ 的强类型 typedef 库}
有一些现成的库可以简化创建强类型整数时的样板代码。Jonathan Muller、Bjorn Fahller 和 Peter Sommerlad 都开发过此类库,还有其他选择可用。
\end{myTip}

\begin{myTip}{练习:考虑使用 =delete 删除有问题的转换}
\filename{图 19. 简单的函数声明}

\begin{cpp}
double high_precision_thing(double);
\end{cpp}

如果用 float 调用上述函数很可能是一个 bug,该怎么办?

\filename{图 20. 删除从 'float' 到 'double' 的有问题的隐式提升}

\begin{cpp}
double high_precision_thing(double);
double high_precision_thing(float) = delete;
\end{cpp}

在 C++11 中,任何函数或重载都可以使用 =delete 来禁用。
\end{myTip}

\begin{myTip}{}
启用 clang-tidy 的 Easily Swappable Parameters 检查\footnote{\url{https://clang.llvm.org/extra/clang-tidy/checks/bugprone-easily-swappable-parameters.html}} check
\end{myTip}

\mySubsectionNoFile{28.2}{资源}

\begin{itemize}
\item
C++ Weekly - 第107集: “=delete 的威力”\footnote{\url{https://youtu.be/aAvjUU0m6AU}}

\item
Adi Shavit 和 Björn Fahller “关联类型的奇特递归模式”\footnote{\url{https://youtu.be/msi4WNQZyWs}}

\item
研究“仿射空间类型(Affine space types)”。

\item
C++Now 2017: Jonathan Müller “类型安全编程”\footnote{\url{https://youtu.be/iihlo9A2Ezw}}
\end{itemize}




