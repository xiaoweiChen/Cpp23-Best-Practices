如果因为 std::unique\_ptr 不适用于使用场景而提供了析构函数,那么必须对其他特殊成员函数进行 =delete、=default 或手动实现。

这个规则最初称为“三规则”,在 C++11 之后称为“五规则”。

\filename{图54. 特殊成员函数}

\begin{cpp}
struct S {
  S(); // default constructor
       // does not affect other special member functions

  // If you define any of the following, you must deal with
  // all the others.
  S(const S &); // copy constructor
  S(S&&); // move constructor
  S &operator=(const S &); // copy assignment operator
  S &operator=(S &&); // move assignment operator
};
\end{cpp}

\begin{myTip}{}
如果不知道如何处理这些特殊成员函数,使用 =delete 是一种安全的方式!
\end{myTip}

当声明带有虚函数的基类时,也应该遵循“五规则”。

\filename{图55. 多态类型中的五规则}

\begin{cpp}
struct Base {
  virtual void do_stuff();

  // because of the virtual function we know this class
  // is intended for polymorphic use, therefore our
  // tools will tell us to define a virtual destructor
  virtual ~Base() = default;

  // and now we need to declare the other special members
  // a good safe bet is to delete them, because properly and safely
  // copying or assigning an object via a reference or pointer
  // to a base class is hard / impossible

  S(const S &) = delete;
  S(S &&) = delete;
  S &operator=(const S &) = delete;
  S &operator=(S &&) = delete;
};

struct Derived : Base {
  // We don't need to define any of the special members
  // here, they are all inherited from `Base`.
}
\end{cpp}

\begin{myNotic}{}
除了使用 = delete,也可以考虑将这些特殊成员函数设为 protected。
\end{myNotic}

\begin{myTip}{练习:实现自己的 unique\_ptr<> 模板}
要完全做对是很困难的。编写测试用例。理解为什么默认生成的特殊成员函数无法正常工作。

额外加分:使用 C++20 的 constexpr 动态分配支持来实现。
\end{myTip}

\begin{myTip}{练习:在代码中查找违反“五规则”的情况}
当定义特殊成员函数时,很可能没有在现有代码中提供一致的生命周期语义。为了评估其影响,可以快速地将缺失的特殊成员函数声明为 = delete;,然后看看哪些地方会出错。
\end{myTip}

\mySubsectionNoFile{40.1}{资源}

\begin{itemize}
\item
C++ Reference: 三/五/零规则 \footnote{\url{https://en.cppreference.com/w/cpp/language/rule_of_three}}
\end{itemize}












