[[nodiscard]] 是 C++ 的一个属性,用于告诉编译器:如果忽略了一个函数的返回值,则应发出警告。它可以用于函数上:

\filename{图 21. [[nodiscard]] 的使用示例}

\begin{cpp}
[[nodiscard]] int get_value();

int main()
{
  // warning, [[nodiscard]] value ignored
  get_value();
}
\end{cpp}

也可以用于类型上:

\filename{图 22. 在类型上使用 [[nodiscard]]}

\begin{cpp}
struct [[nodiscard]] ErrorCode{};

ErrorCode get_value();

int main()
{
  // warning, [[nodiscard]] value ignored
  get_value();
}
\end{cpp}

C++20 增加了提供描述信息的功能。

\filename{图 23. C++20 中带描述的 [[nodiscard]]}

\begin{cpp}
[[nodiscard("Ignoring this result leaks resources")]]
\end{cpp}

我们之前的 divide 示例是 [[nodiscard]] 的一个直接应用。

\filename{图 24. 在 divide 函数上应用 [[nodiscard]]}

\begin{cpp}
import std;

[[nodiscard]] constexpr auto divide(std::integral auto numerator,
                                    std::integral auto denominator) {
  // is integer division
  if (denominator == 0) {
    throw std::runtime_error("divide by 0!");
  }
  return numerator / denominator;
}

[[nodiscard]] constexpr auto divide(auto numerator, auto denominator) {
  // is floating point division
  return numerator / denominator;
}
\end{cpp}

甚至可以用于构造函数:

\filename{图 25. [[nodiscard]] 构造函数示例}

\begin{cpp}
struct Holder
{
  // warn if the result of this constructor is unused
  [[nodiscard]] Holder() = default;

  // bad practice, but exists so that GCC does, in fact,
  // generate a warning for this code below.
  int *p = new int();
};

int main()
{
  // should generate a warning
  Holder();
}
\end{cpp}

\begin{myTip}{练习:制定使用 [[nodiscard]] 的规则}
阅读 Reddit 上的讨论 “支持广泛使用 [[nodiscard]] 的论点”\footnote{\url{https://www.reddit.com/r/cpp/comments/9us7f3/an_argument_pro_liberal_use_of_nodiscard/}}。重新审视自己的类型和函数。哪些返回值应该标记为 [[nodiscard]]?

如果调用这些函数并忽略结果,应该产生编译错误还是警告?

\begin{itemize}
\item
vector.size()

\item
vector.empty()

\item
vector.insert()
\end{itemize}
\end{myTip}

\mySubsectionNoFile{29.1}{资源}

\begin{itemize}
\item
《支持广泛使用 [[nodiscard]] 的论点》\footnote{\url{https://www.reddit.com/r/cpp/comments/9us7f3/an_argument_pro_liberal_use_of_nodiscard/}}

\item
C++ Weekly - 第30集:C++17 的 [[nodiscard]] 属性\footnote{\url{https://youtu.be/l_5PF3GQLKc}}

\item
C++ Weekly - 第199集:C++20 的 [[nodiscard]] 构造函数及其用途\footnote{\url{https://youtu.be/E_ROB_xUQQQ}}
\end{itemize}
















