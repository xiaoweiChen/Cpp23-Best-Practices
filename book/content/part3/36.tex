
与 SFINAE 相比,概念将最终带来更好的错误信息、更短的编译时间,以及远胜于 SFINAE 的代码可读性。如果我们继续以 divide 函数为例,可以回顾之前“优先使用 if constexpr 而非 SFINAE”一章中的 if constexpr 版本。

\filename{图44. 'divide' 函数的 'if constexpr' 版本}

\begin{cpp}
import std;

template <typename Numerator, typename Denominator>
[[nodiscard]] constexpr auto divide(Numerator numerator, Denominator de\
nominator) {
  if constexpr (std::is_integral_v<Numerator> &&
                std::is_integral_v<Denominator>) {
    // is integral division
    if (denominator == 0) {
      throw std::runtime_error("divide by 0!");
    }
  }

  return numerator / denominator;
}
\end{cpp}

可以使用概念将其拆分为两个不同的函数。概念可以在多种不同上下文中使用。此版本在函数声明后使用了简单的 requires 子句。

\filename{图45. 在 'requires' 子句中使用概念}

\begin{cpp}
import std;

// overload resolution will pick the most specific version
template <typename Numerator, typename Denominator>
[[nodiscard]] constexpr auto divide(Numerator numerator, Denominator de\
nominator) requires
        (std::is_integral_v<Numerator>
        && std::is_integral_v<Denominator>) {
  // is integral division
  if (denominator == 0) {
    throw std::runtime_error("divide by 0!");
  }
  return numerator / denominator;
}

template <typename Numerator, typename Denominator>
[[nodiscard]] constexpr auto divide(Numerator numerator, Denominator de\
nominator) {
  return numerator / denominator;
}
\end{cpp}

此版本将概念用作函数参数。C++20 甚至提供了“auto 概念”,这是一种隐式的模板函数。

\filename{图46. 简洁的概念要求语法}

\begin{cpp}
import std;

[[nodiscard]] constexpr auto divide(std::integral auto numerator,
std::integral auto denominator) {
  // is integer division
  if (denominator == 0) {
    throw std::runtime_error("divide by 0!");
  }
  return numerator / denominator;
}

[[nodiscard]] constexpr auto divide(auto numerator, auto denominator) {
  // is floating point division
  return numerator / denominator;
}
\end{cpp}

\begin{myNotic}{}
概念可以定义复杂的约束,包括对成员的期望。本节仅触及了其可能性的皮毛。
\end{myNotic}

\begin{myTip}{练习:了解 C++20 提供了哪些概念。}
像往常一样,cppreference 会通过提供概念列表来帮助你\footnote{\url{https://en.cppreference.com/w/cpp/concepts}}。
\end{myTip}

\begin{myTip}{练习:创建你自己的概念。}
这个例子是否让你想到一些你想要但 <concepts> 头文件并未提供的概念?查看 cppreference 上非常简单的 std::integral 概念的实现\footnote{\url{https://en.cppreference.com/w/cpp/concepts/integral}},看看是否能给你带来启发。
\end{myTip}

\mySubsectionNoFile{36.1}{资源}

\begin{itemize}
\item
C++ Weekly 第194集:从 SFINAE 到 C++20 的概念\footnote{\url{https://youtu.be/dR64GQb4AGo}}

\item
C++ Weekly 第196集:什么是 requires requires\footnote{\url{https://youtu.be/tc0hVIOJk_U}}
\end{itemize}





