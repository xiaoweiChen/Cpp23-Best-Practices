C++20 可以自动生成任意比较运算符(由于其形状,也称为“太空飞船运算符”)。

\begin{itemize}
\item
如果你定义了 ==,编译器将自动生成 !=

\item
如果你定义了 <=>,编译器将生成所有其他比较运算符(但 == 和 != 除外)

\item
可以显式地将比较操作声明为 default
\end{itemize}

\filename{图49. 基本的太空飞船运算符示例}

\begin{cpp}
struct MyData
{
  int i;
  int j;

  // provide all comparisons
  friend auto operator<=>(const MyData &, const MyData &) = default;
};
\end{cpp}

C++ 标准库已开始弃用显式的比较操作,可查阅其示例。

请注意,为了性能考虑,std::string 实现了自定义的 operator==(提供 == 和 !=)和 operator<=>(提供其余比较)。

\url{https://en.cppreference.com/w/cpp/string/basic_string/operator_cmp}

\begin{myNotic}
如果为类型提供了自定义的 operator<=>,编译器将不再为你提供 operator== 或 operator!=!这与特殊成员函数的“零规则”和“五规则”类似。
\end{myNotic}

\begin{itemize}
\item
如果提供了自定义的 operator<=>,则必须自己提供 operator==

\item
如果需要自定义的 operator<=>,那么显式默认的 operator== 很可能无法正确工作
\end{itemize}

\begin{myTip}{练习:实现一个太空飞船运算符}
\filename{图50. 太空飞船运算符练习}

\begin{cpp}
import std;

struct Container
{
  // fixed-capacity container.
  std::array<int, 10> data;

  // size is the number of currently used elements
  std::size_t size;

  // what does the comparison operator need to look like?
  // will a defaulted one work?
  // do we need a custom operator==?
};
\end{cpp}
\end{myTip}

\mySubsectionNoFile{38.1}{资源}

\begin{itemize}
\item
cppreference.com 文档\footnote{\url{https://en.cppreference.com/w/cpp/language/default_comparisons}}
\end{itemize}












