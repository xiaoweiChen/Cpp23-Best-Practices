
优先按以下顺序选择你的容器:

\begin{itemize}
\item
std::array<>

一种固定大小、基于栈的连续容器。数据大小必须在编译时已知,并且你必须有足够的栈空间来容纳数据。这种容器有助于我们优先使用栈而非堆。由于其位置已知且内存连续,std::array<> 成为了一种“零开销抽象”。编译器可以执行额外的一系列优化,因为它知道数据的大小和位置。

\item
std::vector<>

一种动态大小、基于堆的连续容器。虽然编译器不知道数据最终会存放在哪里,但它知道元素在内存中是彼此相邻排列的。这种连续性为编译器提供了更多的优化机会,并且对缓存更友好。
\end{itemize}

几乎其他任何容器的使用都需要添加注释并说明原因。对于小型容器,使用带线性搜索的扁平化映射(flat map)很可能比 std::map 更好。

但不要对此过于狂热。如果你确实需要键值查找,请使用 std::map,然后评估它是否具有你想要的性能和特性。

\begin{myTip}{练习:用 array 替换 vector}
寻找固定大小的 vector,并尽可能将其替换为 array。借助 C++17 的类模板参数推导(Class Template Argument Deduction),这可以变得更加容易。

\filename{图41. 用于固定大小数据的 ‘const std::vector‘。}

\begin{cpp}
const std::vector<int> data{n+1, n+2, n+3, n+4};
\end{cpp}

可以变为

\filename{图42. 用于固定大小数据的 ‘const std::array‘。}

\begin{cpp}
const std::array<int, 4> data{n+1, n+2, n+3, n+4}; // C++11
const std::array data{n+1, n+2, n+3, n+4}; // C++17
\end{cpp}

你已经将它们声明为 const,现在如果可能的话,再进一步将它们改为 constexpr。
\end{myTip}

\mySubsectionNoFile{34.1}{资源}

\begin{itemize}
\item
 Bjarne Stroustrup “Are lists evil?”\footnote{\url{https://www.stroustrup.com/bs_faq.html\#list}}
\end{itemize}







