
C++23 对 C++ 模块提供了完整支持,并且标准库现在被要求提供模块。

\filename{图 26. C++23 模块之前}

\begin{cpp}
#include <string>
#include <vector>
#include <map>

int main()
{
  std::map<int, std::vector<std::string>> data;
  // do stuff
}
\end{cpp}

\filename{图 27. C++23 模块之后}

\begin{cpp}
import std;

int main()
{
  std::map<int, std::vector<std::string>> data;
  // do stuff
}
\end{cpp}

据微软提供的证据显示(在撰写本节时,微软拥有最完整的模块实现),使用模块实际上比包含头文件要快得多。

\begin{myTip}{}
导入一个模块基本上是“零成本”的。
\end{myTip}

当包含一个模块时,无需立即解析代码,因此实际上没有开销。这也就是为什么现在整个标准库可以统一通过单个 import 指令引入。

\begin{myWarning}{}
这是一个非常基础但完整的 C++20 模块定义示例。

有关模块的实际使用,请参阅资源部分获取更多信息。
\end{myWarning}

\filename{图 28. 非常基础的 .ixx 模块接口文件}

\begin{cpp}
// my_module.ixx
export module my_module;

import std;

// I'm only exporting the float overload
export constexpr [[nodiscard]] float calc(float val) noexcept
{
  return val * 10.1f;
}

export void greet(std::string_view name);
\end{cpp}

\filename{图 29. 非常基础的 .cpp 模块实现文件}

\begin{cpp}
// my_module.cpp
module my_module;

import std;

void greet(std::string_view name)
{
  std::println("Hello {}", name);
}
\end{cpp}

\filename{图 30. 非常基础的模块使用方式}

\begin{cpp}
import my_module;

int main()
{
  greet("Jason");
}
\end{cpp}

\mySubsectionNoFile{30.1}{资源}

\begin{itemize}
\item
《模块:初学者指南》- Daniela Engert - Meeting C++ 2019\footnote{\url{https://youtu.be/Kqo-jIq4V3I}}

\item
《当代 C++ 实践》- Daniela Engert - CppCon 2022\footnote{\url{https://youtu.be/yUIFdL3D0Vk}}

\item
《那么,你想跨平台使用 C++ 模块吗?》- Daniela Engert - C++ on Sea 2023\footnote{\url{https://youtu.be/DJTEUFRslbI}}
\end{itemize}














