
一个令人惊讶的复杂性隐藏在看似简单的 C++ Lambda 表达式背后。Lambda 最初在 C++11 中引入,起初功能受限。随着每个 C++ 版本的演进,Lambda 变得越来越灵活和强大。
Lambda 反转了 C++ 其他部分的一些默认规则。它们默认是 const 的,并且在可能的情况下会自动成为 constexpr;这些特性给了我们一些我们希望整个语言都能拥有的东西。

\filename{图 127. Lambda 语法}
\begin{cpp}
lambda-expression:
  lambda-introducer lambda-declarator(opt) compound-statement
  lambda-introducer < template-parameter-list > requires-clause(opt)
       lambda-declarator(opt) compound-statement
lambda-introducer:
  [ lambda-capture(opt) ]
lambda-declarator:
  ( parameter-declaration-clause ) decl-specifier-seq(opt)
        noexcept-specifier(opt) attribute-specifier-seq(opt)
        trailing-return-type(opt) requires-clause(opt)
\end{cpp}

如果你能读懂标准文档中的“行话”\footnote{\url{http://eel.is/c++draft/expr.prim.lambda}},你就可以自己深入探究 C++20 Lambda 的所有特性。

\filename{图 128. 截至 C++20 允许的 Lambda 示例。}
\begin{cpp}
// 有效的空 Lambda,什么都不做
[]{};
// 参数列表是可选的
{};
// C++17 显式 constexpr 和 void 返回
 constexpr -> void {};
// 立即调用的 Lambda
auto i = { return 42; }();
// 在 C++17 之前不允许,因为需要 constexpr
constexpr auto j = []{ return 42; }();
// 泛型 Lambda,C++14
[](auto x){ return x + 42; };
// 可变参数 Lambda,C++14
[](auto ... x){ return std::vector(x...); };
// 值捕获,C++11
i{ return i + 42; };
// 通用捕获,C++14 (这里的 i 是什么类型?)
[i = 42]{ return i + 42; };
// 有状态的 Lambda,C++11
i mutable { return ++i; };
// 显式模板,C++20
[](T x){ return x + 42; };

// C++14 的泛型 Lambda,返回一个 C++20 的 Lambda,
// 该 Lambda 使用可变参数捕获表达式,并返回一个对捕获值进行折叠表达式求和的结果。
[](auto ... val){ return [...val = val]{ return (val + ...); }; };
\end{cpp}

如果你理解了 C++ Lambda 的每一个方面以及编译器如何实现它们,那么你就掌握了 C++ 中所有重要的知识。
这就是为什么我专门在 YouTube 上制作了一个关于 Lambda 的 C++ 课程\footnote{\url{https://www.youtube.com/playlist?list=PLs3KjaCtOwSY_Awyliwm-fRjEOa-SRbs}}。

2018 年,当编译器刚开始支持 C++20 的新 Lambda 时,我使用 Lambda 实现了一个基本符合标准的 std::bind 版本。

\filename{图 129. 使用 C++20 Lambda 实现的 ‘std::bind‘。}
\begin{cpp}
template <std::size_t Idx>
struct Placeholder {};

template <typename T>
struct Bound {
  constexpr decltype(auto) operator()(auto &&...param) const {
    return t(std::forward<decltype(param)>(param)...);
  }

  T t;
};

template <typename T>
Bound(T) -> Bound<T>;

template <std::size_t Idx, typename T>
constexpr decltype(auto) get_param(const Placeholder<Idx> &,
                                   T &&t) {
  return std::get<Idx>(t);
}

template <typename Param, typename T>
constexpr decltype(auto) get_param(Param &&param, T &&) {
  return std::forward<Param>(param);
}

template <typename Param, typename T>
constexpr decltype(auto) get_param(const Bound<Param> &b,
                                   T &&t) {
  return std::apply(b, std::forward<T>(t));
}

constexpr decltype(auto) bind(auto &&callable, auto &&...param) {
  return Bound{
    [callable = std::forward<decltype(callable)>(callable),
    ... xs = std::forward<decltype(param)>(param)]
    (auto &&...values) {
      auto passed_params =
        std::forward_as_tuple(
          std::forward<decltype(values)>(values)...);
      return std::invoke(callable,
                         get_param(xs, passed_params)...);
    }
  };
}
\end{cpp}

我已经有两年没看过这段代码了,但这里有一个 Compiler Explorer 链接供你玩味。

\url{https://godbolt.org/z/hhde3P}

\begin{myTip}{练习:理解并批判性地评价给出的例子。}
对于上面的例子,我本可以做得有什么不同?它能否用概念(concepts)加以约束?是否需要更好的命名?你会怎么做?
\end{myTip}


