\begin{myNotic}{}
我比以往任何时候都更强烈地推荐使用 constexpr!
\end{myNotic}

C++23 和模块(module)结合在一起,几乎消除了所有不使用 constexpr 的理由。

现在,在你的模块接口文件中将函数声明为 constexpr 应该成为默认选择。

\begin{itemize}
\item
有了模块,真正实现了“只为所用付费”(对模块使用者的成本几乎为零)

\item
constexpr 启用了未曾考虑过的编译时计算用例

\item
一个强大的技术是尽可能在编译时完成计算,然后从一个已知的点继续运行时计算

\item
只有当核心库支持 constexpr 时,这些技术才有可能实现。
\end{itemize}

使用 \#define 的时代已经一去不复返了。constexpr 应该成为你的新默认选择!不幸的是,人们常常把 constexpr 想得过于复杂,让我们来分解一下最简单的情况。

\begin{myWarning}{}
需要 C++26 才能从标准库中获得 constexpr 版本的三角函数。
\end{myWarning}

如果看到像这样的代码(我在真实代码中见过):

\filename{图60. 在编译时已知的 'static const'数据}
\begin{cpp}
static const std::vector<int> angles{-90,-45,0,45,90};
\end{cpp}

这真的应该改成:

\filename{图61. 将 'static const' 改为 'static constexpr'}

\begin{cpp}
static constexpr std::array angles{-90,-45,0,45,90};
\end{cpp}

\begin{myNotic}{}
这里的 static constexpr 是必要的,以确保对象不会在每次遇到该函数/声明时重新初始化。static 保证了变量的生命周期与程序相同,并且我们知道它只会初始化一次。
\end{myNotic}

这两者的区别体现在三个方面。

\begin{itemize}
\item
数组的大小现在在编译时就已知

\item
移除了动态内存分配

\item
不再需要支付访问静态变量的开销
\end{itemize}

\begin{myTip}{}
对于头文件中的全局变量,优先使用 inline constexpr 而非 static constexpr,这样链接器可以合并数据结构,减少代码膨胀。
\end{myTip}

主要收益来自前两点,但需要具备 constexpr 的思维才能发现这类机会。也需要 constexpr 的知识来了解,如何在更复杂的情况下对其进行应用。

这种差异可能巨大。

\begin{myNotic}{}
有时,非静态的 constexpr 局部变量可能是最高效的选项,因为数据存在于栈上,而不是二进制文件的其他段中。在优化过程中,请确定这是否重要。
\end{myNotic}

\begin{myNotic}{}
从技术上讲,非静态的 constexpr 变量不一定在编译时计算。然而,它们肯定是在编译时计算的,将其视为在编译时计算是很方便的。
\end{myNotic}

\begin{myTip}{练习:将你的 const 值改为 constexpr}
在阅读代码时,查看所有的 const 值。问自己:“这个值在编译时是已知的吗?” 如果是,要怎么做才能让它变成 constexpr?
\end{myTip}

\begin{myTip}{练习:将 static const 值改为 static constexpr}
浏览你当前的代码库,寻找目前是 static const 的代码。你很可能在某个地方有这种代码。

\begin{itemize}
\item
如果现在是 static const,那么其大小和数据很可能在编译时就已知。

\item
这段代码能变成 constexpr 吗?

\item
是什么阻止了它成为 constexpr?

\item
修改填充这些 static const 数据的函数,让它们也变成 constexpr,需要多少工作量?

\item
记住,如果是头文件中的全局变量,应该优先使用 inline constexpr。
\end{itemize}
\end{myTip}

\begin{myTip}{练习:让头文件中的代码支持 constexpr}
如果有已经在头文件中定义的函数和类型,尝试让这些函数和类型完全支持 constexpr。
\end{myTip}

\begin{myTip}{练习:将头文件中定义的 constexpr 函数移到模块接口文件中}
我们希望避免重复解析 constexpr 函数,而模块接口文件为我们提供了这种方式。
\end{myTip}

\begin{myTip}{练习:寻找更多可以改为 constexpr 的非 IO 函数}
在掌握这项技能的过程中,基本上非 IO 函数都可以改造成 constexpr 函数。

\begin{itemize}
\item
将这些函数移到模块接口文件中。

\item
将这些函数声明为 constexpr。
\end{itemize}
\end{myTip}

\mySubsectionNoFile{43.1}{资源}

\begin{itemize}
\item
C++Now 2017: Ben Deane \& Jason Turner “把所有东西都声明为 constexpr!”\footnote{\url{https://youtu.be/HMB9oXFobJc}}(关于现代 constexpr 技术略显过时)

\item
C++ Weekly 第233集: constexpr map vs std::map \footnote{\url{https://youtu.be/INn3xa4pMfg}}

\item
Meeting C++ 2017: Jason Turner “实用的 constexpr” \footnote{\url{https://youtu.be/xtf9qkDTrZE}}

\item
C++ Russia 2019: Hana Dusíková “编译时常量正则表达式的现状”\footnote{\url{https://youtu.be/r_ZASJFQGQI}}

\item
C++ Weekly 第312集: 停止使用 constexpr(并使用这个代替!)\footnote{\url{https://youtu.be/4pKtPWcl1Go}}

\item
C++ Weekly: constexpr vs static constexpr \footnote{\url{https://youtu.be/IDQ0ng8RIqs}}

\item
C++ Weekly: static constexpr vs inline constexpr \footnote{\url{https://youtu.be/QVHwOOrSh3w}}

\item
cons\_expr: 一个能够在编译时运行的(类似 Scheme)脚本语言 \footnote{\url{https://github.com/lefticus/cons\_expr}}
\end{itemize}





