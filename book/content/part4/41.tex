
如果你发现自己正要选中一段代码并进行复制:请停下来!
后退一步,重新审视一下这段代码。

\begin{itemize}
\item
你为什么要复制它?

\item
源代码和目标代码会有多相似?

\item
将其做成一个函数是否更有意义?

\item
记住,不要害怕使用模板
\end{itemize}

我发现,这条简单的规则对我的代码质量产生了最直接的影响。

如果复制的内容是打算放在当前函数内部,可以考虑使用 lambda 表达式。

C++14 风格的 lambda(使用泛型参数,即 auto 参数)提供了一种简单易用的方法,来创建可重用的代码。它可以在不同数据类型间共享,同时又无需直接处理复杂的模板语法。

\begin{myTip}{练习:尝试使用 CPD。}
有一些不同的“复制粘贴检测器”(copy-paste-detector)工具,可以在你的代码库中查找重复的代码。
在这个练习中,下载 PMD CPD 工具\footnote{\url{https://pmd.github.io/latest/pmd_userdocs_cpd.html}} 并在你的代码库上运行它。
如果你使用 Arch Linux,可以通过 AUR 安装此工具。包名为 pmd;工具名为 pmd-cpd。
你能识别出代码库中哪些关键部分是通过复制粘贴而来的吗?
如果你发现其中一个版本存在 bug,会发生什么?你能否确保所有需要更新的地方都被找到并修复?
\end{myTip}

\mySubsectionNoFile{41.1}{资源}

\begin{itemize}
\item
复制粘贴编程 \footnote{\url{https://www.viva64.com/en/t/0068/}}

\item
最后一行效应 \footnote{\url{https://www.viva64.com/en/b/0260/}}

\item
我不会复制粘贴代码 \footnote{\url{https://twitter.com/bjorn_fahller/status/1072432257799987200}}
\end{itemize}








