SFINAE 有点像“只写”代码。if constexpr 没有 SFINAE 那么灵活,但在能用的时候请使用它。
让我们回顾一下在“在多种情况下优先使用 auto”一节中见过的 divide 例子:

\filename{图 106. C++14 的 divide 模板。}
\begin{cpp}
template<typename Numerator, typename Denominator>
auto divide(Numerator numerator, Denominator denominator)
{
  return numerator / denominator;
}
\end{cpp}

现在,我们希望在进行整数除法时添加不同的行为。在 C++17 之前,我们会使用 SFINAE(“替换失败并非错误”)。这本质上意味着,如果一个函数的实例化失败,它就会从重载决议中被移除。

\filename{图 107. 使用 SFINAE 的 'divide' 函数。}
\begin{cpp}
#include <stdexcept>
#include <type_traits>
#include <utility>

template <typename Numerator, typename Denominator,
          std::enable_if_t<std::is_integral_v<Numerator> &&
                           std::is_integral_v<Denominator>,
                           int> = 0>
auto divide(Numerator numerator, Denominator denominator) {
  // 整数除法
  if (denominator == 0) {
    throw std::runtime_error("divide by 0!");
  }
  return numerator / denominator;
}

template <typename Numerator, typename Denominator,
          std::enable_if_t<std::is_floating_point_v<Numerator> ||
                           std::is_floating_point_v<Denominator>,
                           int> = 0>
auto divide(Numerator numerator, Denominator denominator) {
  // 浮点数除法
  return numerator / denominator;
}
\end{cpp}

C++17 中的 if constexpr 可以简化这段代码:

\filename{图 108. 使用 'if constexpr' 在编译时改变行为的选项。}
\begin{cpp}
import std;

// 注意,我们可以对 numerator 和 denominator 使用 auto
// 但这实际上会通过要求我们使用 decltype 来重新获取类型信息
// 从而使 if constexpr 代码变得复杂。
template <typename Numerator, typename Denominator>
[[nodiscard]] constexpr auto divide(Numerator numerator, Denominator de\
nominator) {
  if constexpr (std::is_integral_v<Numerator> &&
                std::is_integral_v<Denominator>) {
    // 整数除法
    if (denominator == 0) {
      throw std::runtime_error("divide by 0!");
    }
  }

  return numerator / denominator;
}
\end{cpp}

\begin{myTip}{}
在很多情况下,为了提高可读性,请考虑使用 concepts 而不是 if constexpr。
\end{myTip}

\filename{图 109. 使用约束而不是 'if constexpr'}
\begin{cpp}
import std;

// 默认选择的重载
[[nodiscard]] constexpr auto divide(auto numerator, auto denominator) {
  return numerator / denominator;
}

// 仅当两个参数都是整数类型时才会调用此版本
[[nodiscard]] constexpr auto divide(
  std::integral auto numerator, std::integral auto denominator) {

  // 整数除法
  if (denominator == 0) {
  throw std::runtime_error("divide by 0!");
  }

  return numerator / denominator;
}
\end{cpp}

\begin{myNotic}{}
if constexpr 块内的代码在语法上仍然必须是正确的。if constexpr 与 \#define 不同。
\end{myNotic}

\begin{myTip}{}
你通常可能会选择放在 if 块外部的代码,现在可能需要放在 else 块内部,以确保它不会用无效的类型进行实例化。
\end{myTip}

\begin{myTip}{练习:研究“内省式设计”(design by introspection)}
if constexpr、concepts 和 requires 的结合产生了一种非常强大的、几乎类似反射的能力。

\filename{图 110. “内省式设计”示例}
\begin{cpp}
constexpr void add_values(auto &container, auto first, auto second)
{
  // 我们在编译时询问,这个容器
  // 是否具有 reserve 和 size 成员。
  if constexpr(requires {container.reserve(container.size() + 2); }) {
    // 如果这是一个有效的操作,则在运行时执行它
    container.reserve(container.size() + 2);
  }

  // either way, add the values
  container.push_back(first);
  container.push_back(second);
}
\end{cpp}
\end{myTip}

\begin{shell}
该技术最早由 Andrei Alexandrescu 于 2001 年提出,后由 Kris Jusiak 引入到 C++20 中。
\end{shell}

\mySubsectionNoFile{55.1}{资源}

\begin{itemize}
\item
C++ Weekly 特别版:使用 C++17 的 constexpr if\footnote{\url{https://youtu.be/_Ny6Qbm_uMI}}
\item
C++ Weekly - 第 122 集 - 与 optional 和 variant 一起使用 constexpr\footnote{\url{https://youtu.be/2eCV_udkP_o}}

\item
CppCon 2017: Jason Turner “实用的 C++17”\footnote{\url{https://youtu.be/nnY4e4faNp0}}

\item
C++ Weekly - 第 242 集 - C++20 中的内省式设计\footnote{\url{https://youtu.be/sy32kAtsIKg}}

\item
Tony Tables 中的 C++17: constexpr if\footnote{\url{https://github.com/tvaneerd/cpp17_in_TTs/blob/master/if_constexpr.md}}
\end{itemize}






