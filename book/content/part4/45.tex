许多人(如 Kate Gregory 和 James McNellis)都曾多次强调过这一点。将对象声明为 const 有两个作用:

\begin{enumerate}
\item
它迫使我们思考对象的初始化和生命周期,而这会影响性能。

\item
向我们代码的阅读者传达了明确的语义。
\end{enumerate}

顺便提一下,如果是一个静态对象,编译器现在可以自由地将其移至二进制文件的常量区域,这可能会影响优化器的行为。

\begin{myTip}{练习:寻找使用 const 的机会。}
当你浏览代码时,应寻找那些未声明为 const 的变量,并将它们改为 const。
\end{myTip}

\begin{itemize}
\item
如果一个变量不是 const,请思考为什么不是?

\item
使用 lambda 表达式或添加一个具名函数是否能让该值变为 const?
\end{itemize}

\begin{cpp}
const auto data = [](){ // no parameters
  std::vector<int> result;
  // fill result with things.
  return result;
}(); // immediately invoked
\end{cpp}

\begin{myTip}{}
由于返回值优化(RVO),使用 lambda 通常不会增加任何开销,甚至可能提升性能。
\end{myTip}

在这一过程中,你是否将一些 static 变量改为了 const?如果是,请继续进行 constexpr 相关的练习。

\begin{myWarning}{}
对于将要返回的值,使用 const 在某些情况下可能会破坏隐式移动(implicit moves)!
\end{myWarning}

然而,需要注意的是,总体而言,依赖返回值的隐式移动和 RVO 可能会有些脆弱。最好的做法是干脆不要给要返回的对象命名。

\begin{cpp}
ReturnType some_function(int value)
{
  // 如果 result 和 ReturnType 的类型不同,
  // 且存在隐式转换,则会破坏隐式移动。
  // 如果类型相同,则依赖编译器进行 NRVO(命名返回值优化),
  // 但这在包含多个分支的代码中并不总是有效
  const auto result = get_value(value + 42);
  return result;
}
\end{cpp}

\filename{图64. 优先避免命名临时对象}
\begin{cpp}
ReturnType some_function(int value)
{
  // 如果 result 和 ReturnType 的类型不同,
  // 且存在隐式转换,则会触发隐式移动。
  // 如果类型相同,则从 C++17 起适用强制的复制/移动省略
  return get_value(value + 42);
}
\end{cpp}

\begin{myNotic}{}
Clang-tidy 的 No Automatic Move\footnote{\url{https://clang.llvm.org/extra//clang-tidy/checks/performance-no-automatic-move.html}} 分析功能在很大程度上是有缺陷的。它在 NRVO 拷贝省略适用时仍会发出警告,而在真正重要的地方却不会警告!(截至 2022 年 2 月 23 日)。参见此分析:\url{https://compiler-explorer.com/z/a4K76nbhq}。
\end{myNotic}

\begin{myWarning}{}
你可能不希望将类成员声明为 const;这可能会破坏移动构造和移动赋值等关键功能,有时甚至是静默发生的。
\end{myWarning}

\mySubsectionNoFile{45.1}{资源}

\begin{itemize}
\item
CppCon 2014: James McNellis \& Kate Gregory “现代化遗留 C++ 代码”\footnote{\url{https://youtu.be/LDxAgMe6D18}}

\item
CppCon 2019: Jason Turner “C++代码异味”\footnote{\url{https://youtu.be/f_tLQl0wLUM}}

\item
C++ 中 const 或引用成员变量的影响\footnote{\url{https://lesleylai.info/en/const-and-reference-member-variables/}}

\item
C++Now 2018: Ben Deane “易于使用,难以误用:C++ 中的声明式风格”\footnote{\url{https://youtu.be/2ouxETt75R4}}(建立在使 const 更易应用的技术之上。)
\end{itemize}




