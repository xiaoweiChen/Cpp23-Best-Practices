范围视图(Range views)可能会以意想不到的方式保存状态,从而导致代码产生出人意料的求值结果。

\filename{图 86. 重复使用 drop 视图,带有缓存效应}
\begin{cpp}
import std;

int main()
{
  std::list<int> values{1,2,3,4,5,6,7,8,9};

  auto drop_2 = values | std::views::drop(2);

  std::println("{}", drop_2);

  values.erase(values.begin());

  std::println("{}", drop_2);
}
\end{cpp}

你可能会惊讶地发现,上面的代码输出为:

\filename{图 87. 重复使用 drop 视图的输出,使用 list。}
\begin{shell}
> 3, 4, 5, 6, 7, 8, 9
> 3, 4, 5, 6, 7, 8, 9
\end{shell}

然而,如果我们把代码改为使用 std::vector,则会得到预期的输出。

\filename{图 88. 重复使用 drop 视图,没有缓存效应}
\begin{cpp}
import std;

int main()
{
  std::vector<int> values{1,2,3,4,5,6,7,8,9};

  auto drop_2 = values | std::views::drop(2);

  std::println("{}", drop_2);

  values.erase(values.begin());

  std::println("{}", drop_2);
}
\end{cpp}

\filename{图 89. 重复使用 drop 视图的输出,使用 vector。}
\begin{shell}
> 3, 4, 5, 6, 7, 8, 9
> 4, 5, 6, 7, 8, 9
\end{shell}

\begin{myNotic}{}
有些人会因为这类问题而建议你完全避免使用视图。
\end{myNotic}

我们将采取一个更为审慎的方法。

\begin{myWarning}{}
不要重复使用视图!
\end{myWarning}

解决此问题的一个相对简单的方案是:

\filename{图 90. 使用 lambda 表达式来创建视图。}
\begin{cpp}
import std;

int main()
{
  std::list<int> values{1,2,3,4,5,6,7,8,9};

  auto drop_2 = []{ return values | std::views::drop(2); };

  std::println("{}", drop_2());

  values.erase(values.begin());

  std::println("{}", drop_2());
}
\end{cpp}

\mySubsectionNoFile{49.1}{资源}

\begin{itemize}
\item
Denver C++ Meetup: 2023-04 - TylerWeaver - 阅读Ranges(不要只听我说的)\footnote{\url{https://youtu.be/H-CMApmuBoQ}}

\item
C++ Ranges 的优美视图、细节与陷阱 - Nico Josuttis - Meeting C++ 2022 大会主题演讲\footnote{\url{https://youtu.be/O8HndvYNvQ4}}
\end{itemize}