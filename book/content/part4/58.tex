你已经在避免使用堆(heap)并使用智能指针来管理资源了,对吧?!

更进一步,在极少数确实需要使用堆的情况下,请务必使用 std::make\_unique<>()\footnote{\url{https://en.cppreference.com/w/cpp/memory/unique_ptr/make_unique}} (C++14)。

在极少数需要共享所有权的情况下,请使用 std::make\_shared<>()\footnote{\url{https://en.cppreference.com/w/cpp/memory/shared_ptr/make_shared}} (C++11)。

\begin{myTip}{练习:你使用 Qt 或其他某个部件库吗?}
你有没有想过要写一个自己的 make\_qobject 辅助函数?给它赋予你需要的语义,并确保使用 [[nodiscard]]。
无论如何,你都可以将 new 的使用限制在少数几个核心库的辅助函数中。
\end{myTip}

\begin{myTip}{练习:使用 clang-tidy 的现代化修复功能。}
通过 clang-tidy,你可以自动将 new 语句转换为 make\_unique<> 和 make\_shared<> 的调用。记得使用 -fix 选项,在发现问题后自动应用更改。
\end{myTip}

\mySubsectionNoFile{58.1}{资源}

\begin{itemize}
\item
clang-tidy modernize-make-shared\footnote{\url{https://clang.llvm.org/extra/clang-tidy/checks/modernize-make-shared.html}}

\item
clang-tidy modernize-make-unique\footnote{\url{https://clang.llvm.org/extra/clang-tidy/checks/modernize-make-unique.html}}
\end{itemize}









