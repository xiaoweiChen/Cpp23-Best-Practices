我不是“几乎总是使用 auto”的拥护者,但让我问你一个问题:std::count 的返回类型是什么?
我的回答是:“我不关心。”

\filename{图 67. ‘const auto‘}
\begin{cpp}
const auto result = std::count( /* stuff */ );
\end{cpp}

或者,如果你更喜欢:

\filename{图 68. ‘auto const‘}
\begin{cpp}
auto const result = std::count( /* stuff */ );
\end{cpp}

\begin{myTip}{}
使用 auto 可以避免不必要的转换和数据丢失。这与范围 for 循环(ranged-for loops)的道理相同。auto 要求必须进行初始化,这一点和 const 一样,其背后的原因也相同。
\end{myTip}

\filename{图 69. auto 要求初始化}
\begin{cpp}
auto i; // 无法编译
auto i = int; // 无法编译
\end{cpp}

示例:

\filename{图 70. 可能产生昂贵的转换。}
\begin{cpp}
const std::string value = get_string_value();
\end{cpp}

get\_string\_value() 的返回类型是什么?如果它是 std::string\_view 或 const char *,那么在所有编译器上都会产生一次潜在的、代价高昂的转换,且不会有任何诊断提示。

\filename{图 71. 不会产生昂贵的转换。}
\begin{cpp}
// 避免了转换
const auto value = get_string_value();
\end{cpp}

此外,auto 返回类型实际上可以显著简化泛型代码。

\filename{图 72. C++ 98 模板用法。}
\begin{cpp}
// 我们之前“不要害怕模板”中的例子
template<typename Arithmetic>
Arithmetic divide(Arithmetic numerator, Arithmetic denominator) {
  return numerator / denominator;
}
\end{cpp}

这段代码强制我们对分子和分母使用相同的类型(你可以试试看,会得到一些奇怪的编译错误)。

\filename{图 73. 让 C++ 98 模板变得更泛化?}
\begin{cpp}
template<typename Numerator, typename Denominator>
/*返回类型是什么?*/
divide(Numerator numerator, Denominator denominator) {
  return numerator / denominator;
}
\end{cpp}

C++98 无法解决这个问题,但 C++11 可以。

\filename{图 74. C++11 尾随返回类型。}
\begin{cpp}
// 使用尾随返回类型
template<typename Numerator, typename Denominator>
auto divide(Numerator numerator, Denominator denominator)
  -> decltype(numerator / denominator)
{
  return numerator / denominator;
}
\end{cpp}

但在 C++14 中,我们可以完全省略返回类型(记住,跳过 C++11)。

\filename{图 75. C++14 的 ‘auto‘ 返回类型。}
\begin{cpp}
template<typename Numerator, typename Denominator>
auto divide(Numerator numerator, Denominator denominator)
{
  return numerator / denominator;
}
\end{cpp}

在 C++20 中,我们可以使用 auto 函数参数进一步简化这段代码(这会为我们创建隐式模板)

\filename{图 76. C++20 的 ‘auto‘ 参数。}
\begin{cpp}
auto divide(auto numerator, auto denominator)
{
  return numerator / denominator;
}
\end{cpp}

如果对你的应用有意义,可以考虑约束这些参数

\filename{图 77. 受约束的 auto 参数。}
\begin{cpp}
// this should also be constexpr and [[nodiscard]]
std::floating_point auto divide(
std::floating_point auto numerator, std::floating_point auto denominator)
{
  return numerator / denominator;
}
\end{cpp}

\begin{myTip}{练习:熟悉 auto 的类型推导。}
\filename{图 78. 例1: val 的类型是什么?}
\begin{cpp}
const int *get();

int main() {
  const auto val = get();
}
\end{cpp}

\filename{图 79. 例2: val 的类型是什么?}
\begin{cpp}
const int &get();

int main() {
  const auto val = get();
}
\end{cpp}

\filename{图 80. 例3: val 的类型是什么?}
\begin{cpp}
const int *get();

int main() {
  const auto *val = get();
}
\end{cpp}

\filename{图 81. 例4: val 的类型是什么?}
\begin{cpp}
const int &get();

int main() {
  const auto &val = get();
}
\end{cpp}

\filename{图 82. 例5: val 的类型是什么?}
\begin{cpp}
const int *get();

int main() {
  const auto &val = get();
}
\end{cpp}

\filename{图 83. 例6: val 的类型是什么?}
\begin{cpp}
const int &get();

int main() {
  const auto &&val = get();
}
\end{cpp}

\end{myTip}

\begin{myTip}{练习:建立你的实验库}
上面的练习非常适合整理成一系列实验,并保存在你 GitHub 的代码片段(gists)中,正如《C++ 并非魔法》一书中所提到的那样。
\end{myTip}

\begin{myTip}{练习:理解 auto 和模板推导的关系}
理解模板的类型推导规则,以及它与 auto 的关系。阅读 C++ 编程语言标准中的 [dcl.spec.auto] 章节。
\end{myTip}

\mySubsectionNoFile{47.1}{资源}

\begin{itemize}
\item
clang-tidy modernize-use-auto\footnote{\url{https://clang.llvm.org/extra/clang-tidy/checks/modernize-use-auto.html}}

\item
“几乎总是使用 auto”(AAA,Almost Always Auto)\footnote{\url{https://herbsutter.com/2013/08/12/gotw-94-solution-aaa-style-almost-always-auto/}}
\end{itemize}

