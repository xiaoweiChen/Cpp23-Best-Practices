C++20 添加了 <format> 头文件,它提供了 std::format 函数。std::format 接收一个格式化字符串和参数,并返回一个 std::string 对象。

format 具有以下优点:

\begin{itemize}
\item
比 iostreams 编译速度更快

\item
比 iostreams 执行速度更快

\item
比 iostreams 更易读

\item
比 printf 系列函数更类型安全
\end{itemize}

\filename{图56. 简单的 format 用法}

\begin{cpp}
import std;

int main()
{
  const auto result = std::format("Hello {}!", "Jason");
}
\end{cpp}

不幸的是,C++20 的使用场景有些局限,主要机制是格式化为字符串:

\filename{图57. format 与 cout 的用法}

\begin{cpp}
// using include because this is a C++20 example
#include <format>

int main()
{
  std::cout << std::format("Hello {}!\n", "Jason");
}
\end{cpp}

为了解决这个问题,C++23 添加了 <print> 头文件,可以通过模块来使用。

\filename{图58. 使用 std::println}

\begin{cpp}
import std;

int main()
{
  std::println("Hello {}!", "Jason");
}
\end{cpp}

\begin{myTip}{}
std::print 有多个重载版本,以及一个名为 std::println 的辅助函数,会自动在消息末尾添加换行符。
\end{myTip}

\begin{myWarning}{}
这个例子是一个过度复杂的示例!
\end{myWarning}

\filename{图59. 使用 std::print 的重载}

\begin{cpp}
import std;

int main()
{
  std::print(std::cout, "Hello");
  std::println(stdout, " World");
}
\end{cpp}

\begin{myTip}{练习}
学习 std::print 的语法,并开始将 std::cout 代码转换为 print 代码。
\end{myTip}

\begin{myTip}{练习}
使用 clang-tidy 的 modernize-use-std-print\footnote{\url{https://clang.llvm.org/extra/clang-tidy/checks/modernize/use-std-print.html}} 检查器来升级现有的 printf 系列函数。
\end{myTip}

\begin{myTip}{练习}
看看你的 AI 编码助手是否能将 std::iostream 的用法转换为 std::format 和 std::print 命令(已知 ChatGPT 在这方面表现相当不错)。
\end{myTip}















