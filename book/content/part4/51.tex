

(在谨慎的前提下)优先使用 ranges、views 和算法,其次才考虑使用范围 for 循环(ranged-for loops)。

\filename{图 93. 循环时使用 int 与 std::size\_t 的对比。}
\begin{cpp}
for (int i = 0; i < container.size(); ++i) {
  // 哎呀,类型不匹配
}
\end{cpp}

\filename{图 94. 循环时容器不匹配。}
\begin{cpp}
for (auto itr = container.begin();
     itr != container2.end();
     ++itr) {
  //  哎呀,我们大多数人都曾在某个时候犯过这个错误
}
\end{cpp}

\filename{图 95. 范围 for 循环的示例。}
\begin{cpp}
for(const auto &element : container) {
  // 消除了以上两个问题
}
\end{cpp}

\begin{myWarning}{}
在范围 for 循环内部迭代时,切勿修改容器本身。
\end{myWarning}

\begin{myTip}{练习:现代化你的循环}
你的代码中可能还存在一些旧式的循环。

\begin{enumerate}
\item
应用 clang-tidy 的 modernize-loop-convert 检查。

\item
查找那些无法被转换的循环。
\begin{itemize}
\item
无法被转换的循环可能代表代码中存在 bug

\item
无法被转换但又没有 bug 的循环,是很好的简化候选对象
\end{itemize}
\end{enumerate}
\end{myTip}

\mySubsectionNoFile{51.1}{资源}

\begin{itemize}
\item
clang-tidy modernize-loop-convert\footnote{\url{https://clang.llvm.org/extra/clang-tidy/checks/modernize-loop-convert.html}}
\end{itemize}















